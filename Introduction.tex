%!TEX root = 0_QE_Report.tex

\chapter{Introduction}
\label{chapter:Introduction}

%Hantu : I miss a clear problem statement. It's kinda there: a complete egress model is important but doesn't exist yet. So you're going to propose one. But it's not clearly described in the introduction. Also consider a different structure: first explain the background/motivation (egress, crowds, blabla), second explain the shortcomings of existing work (you don't need to go in all the details here. After all this is what you're going to do in your literature review. However, it would be good to summarise existing work and it's shortcomings), third make it very clear which problem you're addressing in your thesis (the need for a complete model)



% What is the problem? The thesis paper topic;
In this thesis, the author proposes to create a multi agent based model of crowds engaging in egress which is based on the idea of humans being serial information processors. The model will take into consideration the most current research in human and crowd behavior and integrate this into a model that can simulate the evacuation of thinking, feeling and forgetful humans from a building. The model considers the entire process of evacuation from the point of start of fire to the point where the last person exits. 
%HANTU : Is there nobody who has done this before? I guess there is, and you will probably be more precise in later parts of the report. However, I wonder whether it would make sense to tone down things in the beginning a little and not talk about novelty in the first paragraph of the report. I think it may be good to briefly discuss the shortcomings of existing literature first before highlighting the novelty of your work.

% The reasons which pushed a student to write his or her thesis paper exactly on this topic; Why is it interesting and important?
Modeling crowds and simulating their behavior and movement has become popular in recent times and is being used for a wide range of applications. One of the most famous recent applications was in animating large crowds in the award winning movie, Lord of the Rings, which used the commercial software called MASSIVE~\cite{Regelous:2011vt}. Another frequent user of crowd simulation systems is civil defense authorities who make use of these simulations to study, evaluate and formulate strategies for controlling crowds and for tackling emergencies that can emerge. The Sydney Olympics made use of crowd simulation software (LEGION)~\cite{Still:2000tp} to test the facilities for their ability to accommodate crowds and emergency evacuations. 

These software that simulate crowds are necessarily very complex because the scenario of a crowd evacuating from a stadium or an airport, is itself a very complex system with lots of interacting elements (including people, fire, escalators, etc.) each of which can cause different complications in the system. However, over the years, many models have been developed and perfected. For example, Still~\cite{Still:2000tp}, while making the aforementioned LEGION model of crowds, conducted extensive surveys and analyses of videos to accurately model the movement of the crowds. Even with all the complexity, detail and meticulousness of these models, many psychologists and sociologists are unconvinced about the efficacy and accuracy of the results produced by these simulations~\cite{Aguirre:2004tn,Torres:2010tj,Sime:1995uu}. This is because even the most popular of these models make certain assumptions about human behavior that stand against evidence obtained over the past few decades through extensive studies in social sciences and humanities~\cite{Torres:2010tj,Sime:1995uu}. Many details of human behavior are wrongly abstracted away, when in fact they actually play a very important role in determining the evacuation dynamics. 

One such example of an abstraction often applied to models of crowds is the perception system used by the modeled humans. More often than not, only a simple visual perception system that perceives all other humans and objects within a certain distance to the agent is used. There are two problems with this approach. The first one is obvious; humans have other methods of observing the environment including aural and olfactory perception. The second problem arises because of a limitation of the human brain. It can only process a limited amount of information at any given point of time. This limitation is something that we come across constantly in our everyday life but which we often fail to notice. For example, while sitting engrossed in reading a book it might take a while before we notice someone calling us. It is also because of this same reason, that we are able to listen and understand someone better if we close our eyes and listen. It is also using this same principle that magicians perform their magic tricks without the audience noticing the trick. 

% The thesis topic preface, or the background information on the thesis paper topic; Why is it hard? (E.g., why do naive approaches fail?)
How is this limitation important in modeling a fire evacuation? This can affect the way in which humans perceive the environment and form their cognitive map and thus their egress route. It can also affect the time taken by a participant to start evacuating because he/she might not know about the fire even if the symptoms are right in front of him/her. This is why it is sometimes mistakenly assumed that people evacuating from a building tend to behave irrationally; they are actually just reacting rationally to the limited information that they have~\cite{Kobes:2009jx,Schadschneider:2008cz,Reicher:2008ep,Torres:2010tj,Paulsen:1984ti,Sime:1983uy}. At the individual level, the difference caused by this can be as little as a few seconds delay in starting evacuation. From the perspective of someone looking at the whole fire evacuation system including the building, fire, fire alarms, people, etc.\ i.e.\ the complex system, this is just a microscopic difference. However, when multiplied over an entire crowd of people that are in a building, the effects can be profound. Recently, experts in various fields have realized the importance of acknowledging that most systems in the world are complex systems and these complexities need to be acknowledged and modeled to have any hope of actually understanding and predicting what happens in the real world. One such expert is the illustrious economist Dr. Brian Arthur who is a vocal critic of the major trend in economics to assume the prevalence and importance of equilibrium~\cite{Arthur:2010uy}. He is of the opinion that systems can never attain an ideal equilibrium because of the complex system of interactions present in any economic system.

Modeling such a complex system is not generally a simple task. One of the most popular methodologies for this is by modeling it as an Agent Based Model (ABM). ABM are generally made up of multiple heterogeneous intelligent entities known as agents. Since each agent's behavior can be specified, this method often allows the modeler to implement psychological or microeconomic theories directly without having to abstract away too many details. This approach is generally very useful because it is easier to study and formulate theories at an individual level; apply this theory to model a real world system populated by many such individuals; and finally watch it work. Higher level patterns that emerge from this micro modeling can be analyzed and studied to learn more things about the system and make predictions.

% Why hasn't it been solved before? (Or, what's wrong with previous proposed solutions? How does mine differ?)
Crowd and evacuation simulation has been in the radar of modeling and simulating experts for the past few decades. However, due to limitations in the technology available, most approaches like lattice gas models~\cite{Takima:2002wr} and flow models~\cite{Henderson:1974ve} have concentrated on abstracting away the details and getting an approximate result in order to get such details as are necessary for that particular application. Advances in hardware have removed many of these constraints. As a result, models have increasingly become more detailed and capable~\cite{Pan:2006vp}. At the same time there have been tremendous advancements in our knowledge of human behavior. However, the majority of the state of the art computational models make assumptions about human behavior without grounding them in the theories and findings from social sciences. Being a very inter-disciplinary field this kind of collaboration is absolutely essential. Another limitation of some models is their tendency to concentrate on a particular aspect of evacuation or a particular phase of evacuation behavior without considering the complete process of evacuation. While this is absolutely necessary, it is also necessary to model the complete process which will be discussed in more detail in Sect.~\ref{LiteratureReview:PsychSummary}. 

\section{Problem Statement}
\label{Intro:ProblemStatement}

Several studies over the past few decades have changed our understanding of human behavior during fire evacuations. Some studies~\cite{Kobes:2009jx,Schadschneider:2008cz,Reicher:2008ep,Torres:2010tj,Paulsen:1984ti,Sime:1983uy} have shown how humans always behave rationally with the limited information that they have and that humans hardly ever panic and behave irrationally. Others have found the importance of groups~\cite{Drury:2009ga} and the effect of stress and time constraints on human behavior~\cite{Ozel:2001tn}. Torres's thesis~\cite{Torres:2010tj} compared and analyzed the effectiveness of various theories in explaining a real life fire scenario. Aguirre~\cite{Aguirre:2004tn} gave an excellent criticism of existing computational models of egress and the shortcomings and strengths of different models. 

However, there is still no computational model of \emph{the entire process} of egress. Pre-evacuation behavior and the search for information are two general characteristics of the fire evacuation process that are very rarely considered in egress simulation models. The few computational models of egress that do consider pre-evacuation~\cite{Pires:2005gs,Klupfel:2003wa} behavior are very simple and abstract away too many details. 

It can be argued that such a model of the entire process is unnecessary and completeness could be achieved by combining existing detailed partial models. However, in such a case it would be difficult to maintain model \emph{coherence}. While modeling the complete process of egress, care has to be taken to ensure that the model is coherent. In other words, entirely different and unrelated methods shouldn't be used to model different phases of evacuation behavior. Using a single central theme, grounded in our understanding of human behavior, provides a logical consistency to the model that is useful in both understanding and using the model. Besides logical consistency, another problem that can arise in combining existing unrelated models is discussed in Section~\ref{MACESPMFServ}.

In this thesis, a computational model for simulating the behavior of humans during fire evacuation is proposed. The entire process of evacuation is taken into consideration. The central idea that humans are serial information processors~\cite{Ozel:2001tn} are used as the basis for creating and explaining the entire process in the proposed model which we call the \emph{Information Based EVACuation~(IBEVAC) Model}.

% What are the key components of my approach and results? Also include any specific limitations. The goals you are going to achieve;% The tasks to complete in order to attain the goals, or the direction of the thesis research development;

\section{Key Contributions and Scope of this Thesis}
\label{Intro:Contributions}

Some of the key contributions of this confirmation report and this thesis will be:
\begin{itemize}
	\item A comprehensive multidisciplinary survey and analysis of current literature on fire evacuation and crowd behavior. The major portion of this work has been completed and will be presented as a key part of this confirmation report.
	
	\item A novel information based perception system that can model the complexities and limitations of the human perception system. This model has been created and implemented and presented at the CyberWorlds 2011 conference and will be discussed in detail in this confirmation report.
	
	\item A novel model of pre-evacuation behavior and the process of information seeking. This phase of evacuation has been proven to exist by social scientists but has been generally ignored in most computational models of egress. A computational model for this phase is proposed and presented in this report but it hasn't been implemented at this stage.
	
	\item A landmark based cognitive map of the environment along with a communication system that approximates the way in which actual humans communicate about their knowledge of the environment. Work on this module has only just started. Only the basics of the model is discussed in this confirmation report. It will be elaborated on during the author's PhD candidature and will be a key part of the final thesis.
	
	\item A complete computational model of the human behavior process during egress, starting from pre-evacuation doubtfulness to escape planning that is firmly grounded in the latest research information available. The framework of the model and most of the details have been developed and are presented in this report. The model has not been implemented completely yet. So only some of the details of the implementation of the model are discussed in this report. 	
\end{itemize}


\section{Organization of the Report}
\label{Intro:Organisation}
 This report is organized as follows. Chapter~\ref{chapter:LiteratureReview} provides a comprehensive review of relevant theories and pre-existing models and also provides a critical analysis of some relevant ones. Having presented the current state of the art, the following chapter provides an overview of the overall Information Based EVACuation (IBEVAC) model architecture. Chapter~\ref{chapter:IBP} then presents the new Information Based Perception Model and some experiments that demonstrate its capabilities and working. Chapter~\ref{chapter:TheRemainingModules} gives an introduction to the proposed structure and working of the remaining modules of the architecture. Finally, Chapter~\ref{chapter:ConclusionAndFutureWork} winds up the report by presenting, in brief, the work that remains to be done and a plan of action for the period of the author's candidature.



