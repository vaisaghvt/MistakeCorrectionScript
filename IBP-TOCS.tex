
%%%%%%%%%%%%%%%%%%%%%%% file typeinst.tex %%%%%%%%%%%%%%%%%%%%%%%%%
%
% This is the LaTeX source for the instructions to authors using
% the LaTeX document class 'llncs.cls' for contributions to
% the Lecture Notes in Computer Sciences series.
% http://www.springer.com/lncs       Springer Heidelberg 2006/05/04
%
% It may be used as a template for your own input - copy it
% to a new file with a new name and use it as the basis
% for your article.
%
% NB: the document class 'llncs' has its own and detailed documentation, see
% ftp://ftp.springer.de/data/pubftp/pub/tex/latex/llncs/latex2e/llncsdoc.pdf
%
%%%%%%%%%%%%%%%%%%%%%%%%%%%%%%%%%%%%%%%%%%%%%%%%%%%%%%%%%%%%%%%%%%%


\documentclass[runningheads,a4paper]{llncs}

\usepackage{amssymb}
\setcounter{tocdepth}{3}
\usepackage{graphicx}
\usepackage{paralist}
\graphicspath{{figures/}}

\usepackage[caption=false,font=footnotesize]{subfig}
\usepackage{multirow}
\usepackage{array}

%
% The latest version and documentation can be obtained at:
% http://www.ctan.org/tex-archive/macros/latex/contrib/subfig/
% The latest version and documentation of caption.sty can be obtained at:
% http://www.ctan.org/tex-archive/macros/latex/contrib/caption/
\usepackage{hyperref}
%\usepackage{fancyhdr}
%\pagestyle{fancy}
%% Enabling the custom headers/footers
%\usepackage{lastpage}   
%\rhead{}
%\lhead{}
%\lfoot{The original publication will be available at www.springer.com.}
%\cfoot{}
%\renewcommand{\headrulewidth}{0.0pt}
%\renewcommand{\footrulewidth}{0.4pt}



\newcommand{\keywords}[1]{\par\addvspace\baselineskip
\noindent\keywordname\enspace\ignorespaces#1}




\begin{document}
%
% paper title
% can use linebreaks \\ within to get better formatting as desired
\title{Modeling and Analyzing the Human Cognitive Limits for Perception in Crowd Simulation}

\titlerunning{Modeling Human Cognitive Limits in Crowd Simulation}

\author{Vaisagh Viswanathan \and Michael Lees}
\institute{School of Computer Engineering, Nanyang Technological University, Singapore\\
\email{vaisagh1@e.ntu.edu.sg, mhlees@ntu.edu.sg}}

\maketitle


\begin{abstract}
One of the major components of Agent Based Crowd Simulation is motion planning. There have been various motion planning algorithms developed and they've become increasingly better and more efficient at calculating the most optimal path. We believe that this optimality is coming at the price of realism. Certain factors like social norms, limitations to human computation capabilities, etc.\ prevent humans from following their optimal path. One aspect of natural movement is related to perception and the manner in which humans process information. In this paper we propose two additions to general motion planning algorithms: (1) Group sensing for motion planning which results in agents avoiding clusters of other agents when choosing their collision free path. (2) Filtering of percepts based on interestingness to model limited information processing capabilities of human beings.
\keywords{Agent-Based Model, Sensing, Crowd Simulation, Motion Planning, Visual Cognition, Group Based Perception, Information, Collision Avoidance}
\end{abstract}


\section{Introduction}
\label{intro}
Crowd simulation is a field that has recently been gaining significant attention because of its usefulness in various applications, ranging from simulation of emergency evacuation to animation of large crowds in movies and games. There are a number of different approaches which are typically applied to modeling of human crowds. These include: flow models~\cite{Klupfel:2005to}, force-based models~\cite{Helbing:2000ef}  and agent-based models~\cite{Luo:2008gj}. All models offer different ways of describing human motion and make different assumptions about how interacting individuals affect one anothers motion.

In this paper we focus on agent-based models of crowds; one key aspect of which is \emph{navigation}. Navigation is defined as the process or activity of accurately ascertaining one's position and planning and following a route. In the context of crowd simulation, navigation is generally considered to be the process of planning a route towards a destination and following this route. We refer to the former as \emph{path planning} and the latter as \emph{motion planning}. The higher level path planning is typically done using A-star or other similar algorithms and deals with the static aspects of the environment. Motion planning is a term borrowed from robotics which originally means detailing a task into discrete motions. In the context of crowd simulation, we use the term motion planning to refer to the task of finding a collision free velocity to get from the current point to the next waypoint in a planned path.

There are a number of existing motion planning methods that can effectively and efficiently calculate trajectories that avoid all collisions for agents, even in very dense environments. For robots and computer games, this might be the ideal goal: perfect, smooth and efficient motion. However, for applications like simulation of emergency evacuations the goal is obtaining realistic motion and not smooth and efficient motion. While we all thrive to be mechanically efficient, this is hardly always the case. There exist, among other things, social norms and limits to mental processing capabilities that prevent individuals from following their ideal preferred path. Also, humans do not necessarily use optimality (in any sense) to determine their preferred path. Our approach is a more naturalistic one~\cite{Klein:2009} in that we feel the navigation models should explicitly consider and model human inadequacies and limitations.

The agent-based models (ABMs) we consider, consist of large numbers of heterogeneous, autonomous entities inhabiting a spatially explicit, partially observable environment; macro-level dynamics are said to emerge through the asynchronous interactions among these entities~\cite{Bonabeau:2002um,Epstein:1999vn}. Each of these individual entities will iterate through a sense-think-act cycle, where agents obtain information from their environment through {\em sensing}, make a decision through {\em thinking} and finally carry out their decision by {\em acting}. In many application areas in which ABMs have been applied, including crowd simulation, the emphasis is generally on describing thought processes accurately via rules. However, sensing is a critical aspect in the modeling process and can greatly impact both the individual and emergent properties of the system. The terms perception and sensing are often used interchangeably in the simulation literature. For clarity in explanation we use the term {\em perception} to define the complete process of obtaining a set of (possibly filtered) percepts from the environment. {\em Sensing}, on the other hand, we define as the process of obtaining raw information from the environment. In this definition, and in our model, sensing is a part of perception. 

Miller's seminal work~\cite{Miller:1956tr} on human cognition revealed two important characteristics of human cognition: 
\begin{inparaenum} 
\item Humans constantly group together similar data into \emph{chunks} of information. 
\item At any given time, a human can only process a limited amount of information. 
\end{inparaenum}
In this paper, we make the assumption that this limited capacity results in humans being attracted towards certain kinds of information, e.g.\ a bright light or a celebrity; this, in turn, results in other information in the environment being unnoticed. By organizing information into chunks, humans are able to use their limited information processing capability more efficiently. This ability can manifest itself in different ways. We assume that during motion planning, humans will process a group of people coming towards them as a single obstacle rather than many individuals. This grouping not only helps the person make use of his limited information processing capacity more efficiently, it also helps him/~her conform to social norms that instruct him/~her that walking through a group of interacting people would be rude.

In this paper, we propose an alternative information based naturalistic perception system, which does not focus
on explicit vision, but rather treats the entire human perception system as an information processing entity. We do eventually plan to extend the use of this \emph{information based perception} for higher level path planning and decision making. However, this is beyond the scope of this paper. The remainder of this paper is organized as follows: Sect.~\ref{LitRev} gives an overview of the existing work; the theoretical basis of the proposed model is explained in Sect.~\ref{theory}; in Sect.~\ref{Results}, we show some simulation results that illustrate the effects of implementing the proposed theory; and finally, Sect.~\ref{Conclusion} concludes this article and gives a brief overview of possible future directions of work.


\section{Related Work}
\label{LitRev}

This section of the paper is divided into two parts: In the first section, we present some of the existing work in motion planning for virtual crowds and in the following section, we present some of the work on whose basis the presented limited information model for agents was developed.

\subsection{Motion Planning}

There are various different approaches to motion planning in agent based models with non-discrete space. The earliest agent based approach to collision avoidance was proposed in Reynolds' seminal paper~\cite{Reynolds:1987vm} on a model of the flocking behavior of birds. There are various simple approaches to modeling human motion like Klein and K\"oster's~\cite{Klein:2009} use of an electric potential based model; positive charges are assigned to goals and negative charges to obstacles and agents. Okazaki and Matsushita~\cite{Okazaki:1993wh} uses a similar approach of using magnetic poles instead of columbic charges. There are also slightly more complicated approaches like the one proposed by Pettr\'e et al~\cite{Pettre:2009fya} which considers the effect of errors in perception on motion planning. In this section, only two of the most popular models for motion planning and collision avoidance used in agent based models, viz. the Social Forces model and the Reciprocal Velocity Obstacle (RVO) model are presented.

The social forces model was first introduced in Helbing's paper~\cite{Helbing:1995ie}. In this model, each agent is modeled as a particle that has multiple forces acting on it. Repulsive forces help in collision avoidance and attractive forces model goal directed and grouping behavior. Over the years, this model has been extended and combined with other higher level behavior models. For example, in~\cite{Kamphuis:2004uu} more complicated group movement was modeled with an underlying social forces model for collision avoidance. In his thesis, Still~\cite{Still:2000tp} criticized the heavily mathematical approach which, according to him, is too complicated to be the natural way in which humans try to avoid crowds.

Another ABM that is increasingly becoming popular for collision avoidance is based on the idea of using the relative motion of objects to determine their time to collision. A velocity is then selected which maximizes this time. This algorithm, based on Reciprocal Velocity Obstacles (RVO) was first extended for use with multi agent systems in~\cite{vandenBerg:2008cq}. Since then there have been several modifications and improvements to the system but the underlying algorithm still remained the same. CLEARPATH~\cite{Guy:2009gu} which mathematically optimized RVO was the first to introduce a change in the underlying algorithm. Later, Guy et al.~\cite{Guy:2010ko} introduced an entirely new approach to RVO that was based on computational geometry and linear programming. They called this new approach \emph{RVO2}. This method further improved the efficiency and smoothness of the system. In~\cite{Guy:2010te}, the introduction of a personal space factor and an observation delay made the algorithm more appropriate for virtual humans.

In~\cite{Guy:2010uv}, Guy et al. introduced an extension to RVO in the form of a higher level navigation based on the principle of least effort. While it is obvious that rational humans would prefer taking the path of least effort, as was explained in Sect.~\ref{intro}, humans do not have perfect knowledge or perfect calculation. Also, it is arguable whether humans are always rational enough to choose least effort as their goal. Another important optimization that was introduced in this paper was using the idea of clustering very distant objects into KD-trees to reduce computational cost. While this might sound similar to the idea that is suggested in our paper, there are two fundamental reasons why this is different from our algorithm: Firstly, we use multiple levels of clustering which will be explained in more detail in Sect.~\ref{ClusteringSection}. Secondly, the motivation and hence design is significantly different: we use clustering as a reflection of how agents perceive their environment and not an optimization for collision avoidance.

There are several works~\cite{Kamphuis:2004uu,Qiu:2008ww} that have tried to model groups in models of crowds. It is important to note that the clustering based group perception proposed in the present paper is distinct from the models of groups in crowds. The groups that are perceived are transient and are just a number of agents which happen to occupy the same location and \emph{appear} to be a group to a perceiving agent. These perceived groups may or may not be the same as actual pre-existing groups.


\subsection{Limits of human perception}
\label{ReviewPerception}
Most motion planning systems focus on optimality of motion. This is either in terms of selecting a path which attempts to avoid collisions with minimal deviation, or in the sense that they are capable of obtaining accurate information about the environmental state. While certain extensions have been previously suggested for making models more realistic, there hasn't been any approach to bring about a human like perception system in multi agent based crowd simulation. 
%As an example and a starting point, in this study, we propose two additions to the perception models used in existing agent based motion planning models.

Miller~\cite{Miller:1956tr} proposed the idea that humans process $7\pm 2$ chunks of information. Cowan~\cite{Cowan:2001wi} argued that humans can actually only cognitively process $4\pm 1$ chunks at any given time. Also, others~\cite{Hochberg:1953eh} have shown that humans try to group together similar information so that information can be encoded in the simplest possible format. This is called the simplicity principle~\cite{Hochberg:1953eh}. Based on these ideas, we propose a method that will emulate how humans perceive groups whenever possible and propose a system in which the agents avoid groups rather than individuals. We refer to this perception system as Group Based Perception (GBP). We have done this using the Evolving Clustering Method (ECM)~\cite{Song:2001vg} and computational geometry based RVO2~\cite{Guy:2010ko}. But our approach can, in principle, use almost any clustering and collision avoidance algorithms. 

Some studies~\cite{Itti:2001wa,OReagan:1999wj,Triesch:2003vz} have shown that humans only pay attention to certain salient features in the objects that they perceive. This results in them not noticing changes in items that are not of interest to them. In~\cite{OReagan:1999wj}, the authors classify elements as either central interest or marginal interest elements and prove that the internal representation of the visual world is rather sparse and essentially contains only central interest information and not information of objects of marginal interest. The factors that influence how interesting a particular object is extensively discussed in~\cite{Itti:2001wa}. In the present paper, we do not propose to model all the complexities of human perception and visual cognition, we would rather like to propose an agent based model for crowds which can not only show a basic implementation of these ideas but can be easily extended when required, to model more complicated visual cognition.

Broadbent~\cite{Broadbent:1965is} has extensively discussed the idea of using information theory for modeling human perception. He introduced various studies that indicate that humans have an upper bound on their capacity for holding information for perception. For a single dimension, this limit is roughly estimated to be about 5-6 percepts. For more than one dimension, the number of discernible alternatives is larger but not as large as would be expected if each dimension was completely independent. The idea of humans being able to process only a limited amount of information is not new to computer animation either. Hill~\cite{Hill:1999ww} was one of the first to introduce the importance of cognition in sensing. Courty et al.~\cite{Courty:2003hy} used a saliency map based approach and Kim et al~\cite{Kim:2005ub} used cost-benefit analysis in a decision theory based approach to determining the interest points. Grillon and Thalmann~\cite{Grillon:2009hf} automated this process of interest point determination. They used criteria like proximity, relative speed, relative orientation and periphery to determine the interestingness of various features. Similar criteria are used in this paper. However, our application being particularly collision avoidance, the information which the agents perceive are dynamic obstacles, i.e.\ other agents or groups of agents. The proposed Group Based Perception system for humans differs from traditional perception systems in that, rather than using a level of perception limited by distance or occlusion of sight, we limit the amount of information, or number of obstacles, which the agents can cognitively process.

\section{The Theory}
\label{theory}

\begin{figure}[!tb]
\centering
\includegraphics[width=\textwidth]{PerceptionAndActing}
\caption{An agent perceives and then acts}
\label{Agent Perception Act}
\end{figure}

This section explains the agent perception system that is proposed in this paper. The sense-think-act cycle of agents was introduced in Sect.~\ref{intro}. Figure~\ref{Agent Perception Act} illustrates how motion planning works in an agent in terms of this sense-think-act cycle. An agent's perception can be described by a function $f: Env\rightarrow p*$, where $p*$ is the set of percepts. Each percept $p$ is then processed by the agent in its decision making process, which in turn will determine an appropriate action for collision avoidance. In our case, the motion planning module is passed a set of percepts which consists of neighboring agents and static obstacles which it processes to find the optimal or most appropriate velocity for reaching the goal. Typically, this list of neighbors is a set of agents within some cone of vision or some distance away from the agent. In this paper we propose a modification to the perception procedure such that it takes place in three phases: \emph{clustering}, \emph{sensing} and \emph{filtering}. Figure~\ref{Agent Clustered Perception} gives an overview of the complete process that is detailed in the following subsections.

\subsection{Clustering}
\label{ClusteringSection}
Central to our information based perception system is the definition of \emph{information units}. In traditional crowd simulation each individual agent or obstacle is considered as a percept, i.e. as an entity which should be processed by the motion planning system. The first assumption of our approach is that percepts can be both individuals and groups of other pedestrians. Whether an individual considers a group or individual is related to the {\em coherence} of the group and also the distance of the perceiving agent from the group. In order to achieve this, we perform a global clustering across the entire environment of agents. We create $n_l$ layers within the environment, each layer identifies and stores groups of a particular size, with increasing layer numbers storing groups of increasing size. The criteria which determines what actually constitutes a group is itself unknown and probably highly dependent on the individual. We make the assumption that only the proximity of the individuals to one another determine whether a collection of people is perceived as a {\em group}. 

For reasons of efficiency we simplify things by performing a single clustering (for each level) for all agents at every time-step, the consequence is that we are implicitly assuming all agents have the same notion of what constitutes a group. In reality this assumption may be too strong, different people may have different criteria for what they perceive as groups. 

While there are various clustering techniques that could be used for grouping agents, we chose to use ECM~\cite{Song:2001vg} because: 
\begin{inparaenum} 
\item It does not require the number of clusters to be predefined and 
\item It can restrict the maximum radius of a cluster. 
\end{inparaenum}
It is also important to remember that this clustering is done dynamically at \emph{each step} and not as a one time calculation of groups.


\begin{figure}[!tb]
\centering
\includegraphics[width=\textwidth]{ClusterProcessing}
\caption{Perception in agents takes place through three stages: 1. Clustering is done at a global level. The dotted line indicates this separation. Agents $a_{0 \cdots n}$ form m clusters $c_{i,{0 \cdots m_{i}}}$ in layer i where $m \leq n$ and  $i \leq K$ where K is the predetermined number of layers. 2. Sensing is the process by which the agents perceive only a subset~($c_{0 \cdots l}$) of these clusters~($c_{0,{0 \cdots m_{0}}}, \ldots , c_{K,{0 \cdots m_{K}}}$). 3. Filtering further reduces the size of this list and models human visual cognition.}
\label{Agent Clustered Perception}
\end{figure}

First the number of clustering layers is decided. In the Fig~\ref{ClusterLayer}, we illustrate information based perception using two layers. The algorithm starts by initializing a single agent as the first cluster, the maximum clustering radius for layer $i$, $r^{i}_{max}$ is fixed (~\ref{firstLayerEq} and~\ref{secondLayerEq}). Each subsequent agent is then compared with every existing cluster to assess its suitability for addition to that cluster. Suitability is determined by the distance of the agent from the cluster. If the agent lies within an existing cluster, it is simply added to that cluster without updating either the radius or the cluster center. Otherwise, the cluster whose center is closest to the agent is determined. If the agent can be added to this cluster, without exceeding the maximum allowed radius for the cluster, then the agent is added to the cluster and the cluster's radius, center and velocity are updated. On the other hand, if adding the agent violates the maximum radius criteria, then a new cluster is created at the location of the agent.

Once this process is completed for layer $i$, the process is repeated for layer $i+1$ until the clusters for all the layers are determined. This process is illustrated figuratively in Fig.~\ref{Agent Clustered Perception}. The clustering function for layer $i$, $cf_{i}$ allocates one and only one cluster for each agent in each later. This can be represented mathematically as shown below:

\begin{equation}
   \forall a_{k} {\in} A \mbox{ }\exists j \in  [1 , m] \quad cf_{i} : a_{k} {\rightarrow} C_{ij} \mbox{ where } 1 {\leq} m {\leq} n
\end{equation}
\begin{equation}
  \\ \forall a_{j} \in A \quad C_{0j} = a_{j}
\end{equation}  
\begin{equation}
 \\r^{1}_{max} = 2 \alpha * a_{r}
  \label{firstLayerEq}
\end{equation}
\begin{equation}
  \\ \forall i \geq 2 \quad   r^{i}_{max} = 2 \alpha * r ^{i-1} _{max}  
   \label{secondLayerEq}
\end{equation}

Here $a_{r}$ is the average radius of an agent\footnote{In the experiments in this paper, it is assumed that all agents have the same radius. Hence, the radius of every agent is the same as the average radius.} in $A$ which is the set of all agents; $C_{ij}$ indicates cluster $j$ in layer $i$; $m$ is the number of clusters and $n$ is the number of agents. $\alpha$ is a parameter that determines the size of clusters and the range of each region (Fig.~\ref{ClusterLayer}). Through experimentation we found the most pleasing results with $\alpha = 2$.

% Mike, will need a better way to put this. Is it too much to put a figure here?
To correct certain undesirable behavior produced by ECM clustering, a modification was made to the algorithm. With large values of $r_{max}$, there is a chance that distant agents might be grouped into sparse clusters. To counter this problem, we define a \emph{checking circle} as a circle of radius $2 \alpha a_{r}$. If there are no agents within this checking circle, then the cluster is considered sparse and the cluster is removed. The sparseness check is done five times. First with the circle centered at the center of the cluster and subsequently with the checking circles centered at a distance equal to half the distance from the center of the cluster along each of the coordinate axes.

\subsection{Sensing}
\label{perceptionDescription}

Once the agents have been clustered, the next step is to make use of these clusters for motion planning. As previously explained, existing motion planning algorithms need a list of nearby agents and obstacles to determine the most appropriate velocity. The sensing module of our proposed perception mechanism uses the set of $n_l$ layers created in the clustering module. The list of things to avoid will now consist of agents, obstacles and groups of agents. This list of nearby objects is now calculated from the multiple clustering layers as shown in Fig.~\ref{ClusterLayer}.

\begin{figure*}[!tb]
\centering
\includegraphics[width =\textwidth]{AttemptedClusterLayer}
\caption{The figure illustrates how the opaque agent senses objects using 2 clustering layers. The bottom layer is the original environment and the two planes above show the two clustering layers. Clusters in layer 2 are generally bigger than in layer 1. Solid lined circles indicate the normal agents and the clustered agents. The dotted lines show the regions of perception.}
\label{ClusterLayer}
\end{figure*}

From each cluster layer (explained in Sect.~\ref{ClusteringSection}) a ring shaped {\em perception region} $pr_i$ is defined for each agent. This region can be considered as a modification of the sensor range which is used in most ABM.  In the first region ($pr_0$), immediately surrounding the agent performing the sensing, the agent perceives other individual agents from the clustering layer 0. This region extends to a distance $r_{pr_0} = 2\alpha *a_{r}$ from the agent's current location. For each subsequent region, the ring shaped region of sensing is from the boundary of the previous layer's region to the boundary of a circle of radius $2\alpha$ times the radius of the preceding region. So for region $pr_1$ the agent perceives groups of maximum size $r^{1}_{max}$ (Fig.~\ref{ClusterLayer}). If the nearest edge of their minimum enclosing circle is within a distance $d$, such that $ r_{pr_0} < d \leq r_{pr_1}$. The result is a list of obstacles which consists of clusters of various sizes and individual agents.

\subsection{Filtering}

As explained in Sections~\ref{intro} and~\ref{ReviewPerception}, a human being does not cognitively process every single object or obstacle that is within its vision. In other words, an agent can only process a limited amount of information. The information that is processed will be that which is deemed most interesting or important to the agent. So each object in the list obtained from perception is assigned an interestingness score of between 0 and 1 (1.5 for exceptional cases). During the sensing process each agent is given an \emph{information limit} $a_{IL}$, indicating the total amount of information that can be processed by the agent. This limit is a parameter than can change as the stress level or other characteristics of the agent changes~\cite{Ozel:2001tn}.

For this paper we assume that interestingness of an object depends on two criteria:
\begin{inparaenum}
 \item The distance of the object from the agent. 
 \item The angle that the object currently forms with the direction of motion of the agent. 
\end{inparaenum}
A third factor indicating the innate interestingness of the object being perceived can also be used; this can represent a lot of other properties related to interestingness. For example, an object's speed, color, action or something more subjective, i.e.\ it is of interest only to this agent because of certain properties of the agent. For e.g., for a thirsty agent, a water cooler would be interesting whereas it is unlikely to catch the attention of someone else. A more exact definition of interestingness is not the focus of this paper, but the general model here should be able to adapt to more sophisticated definitions.

Based on these criteria, a score is given to each agent. A distance score of 1.5 is given if the distance between two agents is less than or equal to zero. This is to ensure that in high density scenarios where a collision does occur, a collision recovery mechanism is forced on the objects regardless of what angle or how interesting the object is. For other distances the following equation is used to calculate the score for a distance $d$. $\gamma$ and $k$ are parameters which were fixed at 5.0 and 1.11 respectively to get a curve as in Fig~\ref{DistanceScore}.
\begin{equation}
    S_d = max(min(1.0, e^{\gamma / d} - k),0.1)
\end{equation}

\begin{figure}[!tb]
\centering
\includegraphics[width=3.25in]{distanceScore}
\caption{This graph shows the variation of distance score with distance (in metres) used in experiments. A score of 1.5 if a collision has already occured, a score of 1 if it is within 7m and an exponentially decreasing score beyond that distance}
\label{DistanceScore}
\end{figure}

An angle score of $1.0$ is given to all objects forming an angle of less than $a_{min}$ with the agent's direction. For all agents that form an angle of more than $a_{max} $ with the agent's direction,  a score of $ (1-\beta) $ is given. For our experiments a $\beta$ value of 0.9 was used and this is illustrated in Fig.~\ref{AngleScore}. For all angles in between, the angle score linearly decreases to $ (1-\beta) $ from $1$. This is assigned based on the following equation (Fig.~\ref{AngleScore}). All angles are in radians:
\begin{equation}
  S_{\theta} = 1.0 - (\beta * (a - a_{min} ) /  (a_{max}-a_{min}))
\end{equation}

\begin{figure}[!b]
\centering
\includegraphics[width=3.25in]{angleScore}
\caption{This graph shows the variation of angle score with the angle(in radians) formed by the object with the agent used in experiments. For objects forming an angle of less than $70^{\circ}$ (viewing angle $140^{\circ}$, a score of 1 is given. For objects forming an angle of up to $90^{\circ}$, the score linearly decreases to $0.1$ which is the angle score for all remaining obstacles.}
\label{AngleScore}
\end{figure}

The final score for the object is calculated as the product of the $S_{\theta}$ and $S_d$ (as long as distance score is not 1.5). This list of objects is then sorted on the basis of the score that is determined. Objects are then removed from the head of this list in turn and added to the final list of perceived objects as long as the cumulative score of all the perceived objects does not exceed the information limit for the agent, $a_{IL}$. All the remaining objects are dropped from the list of objects sensed and the final list of percepts $p*$ is obtained. In case two objects have the same score, the objects that are moving towards the perceiving agent are given precedence, subsequently closer objects are given preference.

For the implementation in this paper we pass the shortened neighbor list to RVO2~\cite{Guy:2010ko} for calculating the velocity at each time step. Our hypothesis is that the 3-step perception process proposed by us in this paper provides an improvement in two ways: Firstly, there are fewer neighbors and hence, fewer constraints for a given sensor range. Secondly and more importantly, more human like results can be obtained as will be illustrated in Sect.~\ref{Results}.

\section{Results}
\label{Results}

We are currently working towards gathering real world data that would ideally be used for validation of the proposed model. Nevertheless, in the following sections, we use the ideas introduced in Sects.~\ref{intro} and~\ref{ReviewPerception} as the basis for validating different aspects of the proposed model. Two quantitative measurements are used to analyze the model: \emph{Effort Expended} and \emph{Inconvenience Cost}. In proposing their least effort based approach to motion planning~\cite{Guy:2010uv}, Guy et al. used a measure of effort expended to demonstrate the usefulness of their model. This effort was calculated as follows:

\begin{equation}
E = m \int \! (e_s + e_w |\vec{v}| ^2) \, \mathrm{d} t \footnote{$e_{s} = 2.23 \frac{J} {Kg s}$ and $e_{w} = 1.26 \frac{Js} {Kg m^{2}}$ for an average human~\cite{Whittle:2006vsa}}
\end{equation}

In this paper, we use the same measure of effort to analyze and validate our model. For simplicity, we take all agents to have the same average mass of 70~Kg. However, this only measures the mechanical effort involved. To measure the amount of effort spent in decision making, we introduce \emph{inconvenience cost}. The inconvenience cost is the number of time steps in which the agent chose a velocity other than its preferred velocity i.e., the number of times they have to avoid a collision. 

We consider four different scenarios which we consider to be a good way to evaluate the overall performance. First, we simply demonstrate the effects that Group Based Perception can have on the trajectory of an agent both in visual and in quantitative terms. Next, we present the benefits of using a multiple layers of clustering. Following this, we conduct an experiment to demonstrate how group based perception is essential if we are to model a human being's information processing limits. In the final experiment we analyze the effects

\subsection{Group Based Perception}
\label{GBP}

\begin{figure}[!tb]
  \centering
   \subfloat[Traditional Circular Sensor Range]{\label{Exp1_RVO}\includegraphics[width=\textwidth]{Exp1_RVO}}
  \\
   \subfloat[Group Based Perception]{\label{Exp1_GBP}\includegraphics[width=\textwidth]{Exp1_GBP}}
  \label{Exp1}
   \caption{Experiment 1: The effect of Group Based Perception. It can be observed that when Group Based Perception is used, the perception algorithm helps generate motion that avoids entire groups.}
\end{figure}


In this experiment we compare the results of using RVO2 with a traditional simple circular sensor range against RVO2 with a Group Based Perception system. The intention is to show the effect of perceiving agents as groups. Our hypothesis is that by perceiving groups as obstacles the simulation will generate more visually natural motion. In Fig.~\ref{Exp1}, there is a single black agent moving towards the right, and a number of groups of red agents moving towards the left. The black trail shows the path that is taken by the black agent. It can be seen that in Fig.~\ref{Exp1_RVO} where GBP was not used, the agent walked through other groups. Since RVO2 enforces each agent to do half the work to avoid collision, the agents within the group individually give way through its center for the oncoming agent to pass. 

At present we base our argument on the discussion in Sects.~\ref{intro} and~\ref{ReviewPerception}, due to social norms and the human tendency to group information together people generally try to move around an entire group rather than walking directly through a group. As shown in Fig~\ref{Exp1_GBP} our perception algorithm is capable of generating motion which avoids entire groups.

\begin{table}[bp]
\caption{Quantitative analysis of Group Based Perception: Group Based Perception ensures the least inconvenience to majority of agents while ensuring that hardly any more effort is expended.} 
\begin{tabular}{>{\centering}p{1.1in}>{\centering}p{0.85in}>{\centering}p{0.85in}>{\centering}p{0.85in}>{\centering}p{0.85in}}
\tabularnewline 
\hline\hline %inserts double horizontal lines
\multirow {2}{*}{Agent Considered} & \multicolumn{2}{c}{Effort ($* 10^5$)} & \multicolumn{2}{c}{Inconvenience Cost}\\
 & Without GBP & With GBP & Without GBP & With GBP 
 \tabularnewline
\hline
Black Agent  & 71730 & 71726 & 120 & 148 \tabularnewline
All other agents (average) & 1884 & 1880 & 14.28 & 6.56 \\
\tabularnewline
\hline
\end{tabular}
\label{tab:Exp1_QuantitativeAnalysis}
\end{table}

An analysis of the effort expended and the inconvenience cost gives some interesting results. Since the simulation is executed for a given number of time steps, the effort expended is normalized with the progress towards the agent's goal. This is to avoid slow or stationary agents from being considered to be more efficient despite traveling a lesser distance. On comparing the normalized effort in the two scenarios of the black agent, it is found that despite having a much longer path, the GBP enabled agent expends slightly lesser (practically the same) amount of effort than the other. This is because the non-GBP agent has to slow down to wait for the other agents to give way before it can proceed and thus progresses less towards the goal. 

The inconvenience cost comparison gives another interesting, though not surprising, result. The inconvenience cost to the black agent of using Group Based Perception is higher because of the more indirect path that it has to take. However, the average inconvenience caused to all the other agents is significantly lesser. This conforms with the general human reluctance to inconvenience others. It also gives the interesting idea that even if the same amount of mechanical effort is expended in following two different paths, the amount of decision making required for each path might be significantly different.

\subsection{Effects of multi layered clustering}

\begin{figure}[!t]
  \centering
   \subfloat[Using traditional sensing]{\label{Exp2_RVO}\includegraphics[width=6cm,height=10cm]{Exp2_RVO}}
   \hspace{1pt}
  \subfloat[Using Group Based Perception]{\label{Exp2_GBP}\includegraphics[width=6cm,height=10cm]{Exp2_GBP}}     
  \caption{Experiment 2: Effect of multi layered clustering. The multi layered clustering ensures that agents do not try to unrealistically avoid large groups. If the agent is too close to a large group, i.e. it is impossible to go around it, the agent decides to go through it. The underlying RVO2 mechanism ensures that this transition happens smoothly.}
  \label{Exp2}
\end{figure}

In this experiment, we studied the simple scenario where a single (black) agent had to get past a big group of agents to get to its goal. The same experiment was performed by keeping the agent at different distances from the group. The objective of this experiment is two-fold. Firstly, it demonstrates the importance and the working of the multi-layered clustering (Sect.~\ref{perceptionDescription}) used. Secondly, it demonstrates that when agents are very close to each other, where RVO2 already performs well, the Group Based Perception does not interfere with RVO2's functioning.

To recap, the multiple layers are used to describe groups of varying size at varying ranges of perception. This means agents will perceive other agents as groups or individuals depending on the distance; as an agent moves towards a group it will start to perceive the group as individual agents. 

When GBP isn't used, the path followed does not change significantly with distance. The agents in the path of the black agent, give way to the agent, and the black agent just proceeds straight through the center of the large group (Path A in Fig.~\ref{Exp2_RVO}). In the last few cases (Paths B and C), the path is slightly different because the black agent does not have enough time to plan for a smooth, straight path and hence there is a slight deviation. Also, similar to the experiment in Sect.~\ref{GBP} it is forced to slow down in the process.

The result produced when GBP is used is more varied. Four distinctly different paths (labeled D, E, F and G in Fig.~\ref{Exp2_GBP}) are produced based on how far the oncoming black agent is from the big group. At distances between x-y m away, the agent has enough time to perceive the group and avoid it completely (Path D). At distances between y-z m away, due to the size of the group, the agent gets too close to the group such that it then perceives the group as individuals. At this time (as described in Fig.~\ref{ClusterLayer}) the agent performs motion planning on all the individual agents and as a consequence moves through the group, shown by path E. Path F is obtained in a similar fashion; however, the black agent is too close to the group to discern the effect of GBP. At distances closer than this, the path followed by the agent (Path G) is exactly the same as that followed by the agent not using Group Based Perception (Path C). We argue that this type of flexibility in the perception of groups is critical to creating more natural behavior, humans will adapt what they perceive based on success or failure of their attempt to avoid larger groups. 

\begin{figure}[!t]
  \centering
   \subfloat[Black Agent- Effort]{\label{Exp2_Black_Effort}\includegraphics[width=12cm,height=7cm]{Exp2_Black_Effort}}
  \hspace{1pt}
    \subfloat[Black Agent- Inconvenience]{\label{Exp2_Black_Inconvenience}\includegraphics[width=12cm,height=7cm]{Exp2_Black_Inconvenience}} 
    \caption{Experiment 2: Quantitative Analysis - Black Agent. There is a sudden spike in inconvenience cost in the graph because it realizes mid way through trying to avoid the group that it can't actually avoid the group and redirects itself to go through the group instead.}
  \label{Exp2_QuantitativeAnalysisBlack}
\end{figure}          
  % \\
  \begin{figure}[!t]
  \centering
  \subfloat[Remaining Agents- Effort]{\label{Exp2_Remaining_Effort}\includegraphics[width=12cm,height=7cm]{Exp2_Remaining_Effort}}
  \hspace{1pt}
  \subfloat[Remaining Agents- Inconvenience]{\label{Exp2_Remaining_Inconvenience}\includegraphics[width=12cm,height=7cm]{Exp2_Remaining_Inconvenience}}   
  \caption{Experiment 2: Quantitative Analysis - Remaining Agents. Two things to be noted is that the curves merge when the black agent is very close to the group. The only time inconvenience cost is higher is when the black agent changes its plan midway through trying to avoid the larger group.}
  \label{Exp2_QuantitativeAnalysisRemaining}
\end{figure}

Figures~\ref{Exp2_Black_Effort} and~\ref{Exp2_Remaining_Effort} show a comparison of the effort expended by the black agent and the average effort expended by all the remaining agents while using a traditional sensor range and Group Based Perception. As in the previous experiment (Sect.~\ref{GBP}) there is hardly any difference in the effort expended in both scenarios (except for a slight increase for trail E). However, an interesting pattern can be observed in the inconvenience measurement. Firstly, the inconvenience for the rest of the group, is always lesser when GBP is used and almost the same for the black agent when path D is followed. However, when path E or F is followed there is a spike in the inconvenience curve. This can be explained by considering the fact that in both path E and F, the black agent changes its planned path suddenly and decides to go through the group, thus not only increasing its own inconvenience but also the inconvenience caused to others in the group who have to move to give way to the agent. Finally, when path G is followed both the effort and inconvenience count are exactly the same as for path C.

\subsection{Filtering necessitates Group Based Perception}


\begin{figure}[!t]
  \centering
  \subfloat[Initial Scenario]{\includegraphics[height=4cm]{Exp3_initial}}
   \\
   \subfloat[Complete knowledge, without GBP]{\label{Exp3_full_RVO}\includegraphics[width=6cm,height=3cm]{Exp3_full_RVO}}
  \hspace{1pt}
    \subfloat[Info limit of 4, without GBP]{\label{Exp3_4_RVO}\includegraphics[width=6cm,height=3cm]{Exp3_4_RVO}}           
  \\
  \subfloat[Info limit of 4, with GBP]{\label{Exp3_4_GBP}\includegraphics[width=6cm,height=3cm]{Exp3_4_GBP}}
  \hspace{1pt}
  \subfloat[Info limit of 1, with GBP]{\label{Exp3_1_GBP}\includegraphics[width=6cm,height=3cm]{Exp3_1_GBP}}   
  \caption{Experiment 3: The necessity of Group Based Perception. If the cognitive limit on information processing is imposed without considering group based perception, unnecessary collisions and highly irregular trajectories will be produced.}
  \label{Exp3}
\end{figure}

In Sects.~\ref{intro} and~\ref{ReviewPerception}, the fact that humans have limited information processing capacity was explained. In this experiment, we demonstrate that if we are to model a human being's limited information processing capability, it is also necessary to use Group Based Perception. This is done by observing the simple scenario of an agent moving towards two groups of other agents (Fig.~\ref{Exp3}). When no information limit is imposed on the agent, and a normal circular sensor range is used, the agent, as expected, follows a nice straight path through the center of the group. However, when an information limit of $a_{IL} = 4$ is imposed on the agent, the black agent, does not perceive all the individual agents in the group and  as a result it is forced to reconsider its path mid-route. As a result, the irregular trail shown in Fig.~\ref{Exp3_4_RVO} is obtained. However, in the same situation, when Group Based Perception is used, the agent smoothly avoids the whole group (Fig.~\ref{Exp3_4_GBP}). In fact, this smooth path is obtained for as low a limit as $a_{IL} = 1$ (Fig.~\ref{Exp3_1_GBP}).  

\subsection{Effect of filtering of percept information}

\begin{figure}[!tb]
  \centering
   \subfloat[Initial Scenario]{\includegraphics[height=4cm]{Exp4_initial}}
   \\
   \subfloat[InfoLimit 3: Step 1]{\label{Exp4_3_1}\includegraphics[width=6cm,height=3cm]{Exp4_3_1}}
  \hspace{1pt}
    \subfloat[InfoLimit 5: Step 1]{\label{Exp4_5_1}\includegraphics[width=6cm,height=3cm]{Exp4_5_1}}           
  \\
  \subfloat[InfoLimit 3: Step 2]{\label{Exp4_3_2}\includegraphics[width=6cm,height=3cm]{Exp4_3_2}}
  \hspace{1pt}
  \subfloat[InfoLimit 5: Step 2]{\label{Exp4_5_2}\includegraphics[width=6cm,height=3cm]{Exp4_5_2}}   
  \caption{Experiment 4: Effect of filtering of percept information. Higher information capacity results in smoother trajectories.}
  \label{Exp4}
\end{figure}

The final experiment (Fig.~\ref{Exp4}) demonstrates the effect of filtering, i.e.\ having limits on the information processing capabilities of the agents. The scenario consists of an agent moving towards a collection of individuals (moving towards the agent) followed by a group of agents behind the set of individuals. In the first case we set an information limit of $a_{IL} = 5$  so that the agent is continually capable of perceiving a larger number of other agents and groups.  In the second scenario we use a lower limit of $a_{IL} = 3$ such that the agent isn't initially capable of perceiving the group behind the individuals. Figure~\ref{Exp4_5_1} shows how agents perceive the cluster that is farther away, even when there is an immediate collision to avoid. Figure~\ref{Exp4_5_2} shows that the agent manages to move around this group because it had a head start in planning - i.e.\ , it considered the group early when avoiding collisions. In the second scenario we gave a much lower information limit, such that it could process a maximum of 3 or 4 percepts at any given time. Due to this, as seen in Fig.~\ref{Exp4_3_1}, the agent cannot see beyond the immediate obstacles in front and does not prepare in advance to avoid the larger group. Once the agent finally perceives this group, it is too late to move around this group as it perceives the group as individuals and then moves through the group as in Fig~\ref{Exp4_3_2}. 

This experiment illustrates how small differences in the information limit can generate different forms of behavior in the agents~\footnote{Interestingly, the info limit of 3 and 5 correspond to Cowan's~\cite{Cowan:2001wi} finding that all humans can cognitively process only 3-5 chunks of information at any given time}. Clearly the value of the limit is critical to behavior, we also propose that this limit will change with personal characteristics and the emotional state of the agents. In fact we feel that this varying limit of perception is an important factor for collisions in crowds, this is especially relevant in emergency egress scenarios where stress and collisions are critically important to safety planning. We plan to attempt to quantify this information limit through experimentation in future work.

\section{Conclusion}
\label{Conclusion}
In this paper, we have proposed a model of perception based on perceived information rather than spatial distance. We argue that this is a more appropriate model of human perception for crowd and egress simulation. We have illustrated the behavior of this system through experiments and have shown and argued that this creates more realistic group avoidance behavior. We also presented a perception model which incorporated the idea that humans have limited perception capacity such that they only process certain obstacles more relevant to collision avoidance, which in turn will result in a reduction in efficiency of collision avoidance. Critical to the model is the quantification of information limits and appropriate definitions of interest; we plan to conduct real world experiments to attempt to quantify these parameters.

A more naturalistic system like this is essential for the development of an accurate model of crowd evacuation in emergencies. In emergency situations, according to~\cite{Ozel:2001tn}, humans start perceiving cues in the environment differently. In the future, we plan to extend this model by first adding more features to the measure of interestingness score and also by modeling different cues and their effect on the agent's information processing capabilities as suggested in~\cite{Kuligowski:2009un}. The third criteria which we mentioned in Section~\ref{theory}, i.e.\ the inherent interestingness of the object has not been elaborated on in this paper. Also, in this paper we have not implemented any memory for the agent. To accurately simulate virtual humans' and their motion, the fact that they can remember the positions of objects should also be taken into consideration. Virtual humans can also extrapolate the movement of agents that they have perceived previously but are not in their field of vision at the current time. In this paper, we have considered the effect that perception can have on cognition. However, as mentioned in papers like~\cite{Hill:1999ww} there is also a reciprocal effect of cognition on perception where agents would turn towards objects of more interest, we plan to incorporate this in future versions of the model.


\section*{Acknowledgment}
This research has been funded by the NTU SU Grant M58020019


\bibliographystyle{splncs}
\bibliography{tocs}

\end{document}








